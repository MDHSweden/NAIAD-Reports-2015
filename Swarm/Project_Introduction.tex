

\section{Project introduction}
This year the project course has four different projects. ROARy, Butler, Swarm and a UAV project. 
All the students has been divided into pool groups with different specializations. Hardware, which developed all the electronics. Software, they developing and implemented code and simulations for different systems. Communications which worked with all type of communication e.g. how to communicate with the Naiad robot and how to control it. Mechanics they worked with the different structures and CAD programs for all projects.
The working hierarchy that has been used are four different project managers and four pool group leaders, with a different amount of workers. 
The project managers shall treat the pool groups as there was small consult companies and that the project leaders buy a service of them.
It is up to the group leader at every pool to distribute the work among his teammembers in order to achieve all the goals that the project managers has set up.

The purpose of this hierarchy is two folded. Firstly, reuse solutions between the projects in order not to invent the wheel all the time. Secondly, this hierarchy has the possibility to be more dynamic than if all students where locked in a small group working on a specific project.\\

This report will only address the swarm project.

The swarm project consists of two sides. One side consists of the old Naiad platform that was started to be developed in 2013 and the second side is swarm behavior. On the Naiad platform a light communication system for underwater purposes has started to be developed and a Wi-Fi solution for programming the robot has been installed, instead of the old ethernet cable.
A second robot has been started to be developed, which is called Naiads sister. The hull and motors are bought, but not assembled. A design for the hull to the new thrusters has been initiated.

The aim for the swarm behaviour side of the project has been on finding an Algorithm for shape forming, the project members has initiated  the implementation one. 

A new land based platform has been bought called kilobots. The purpose for using a land based platform instead of only the underwater platform, is that the school does not possess a large enough pool on the campus area that can fit more than one robot of the naiad type.

\subsection{The structure of the report}
The rest of the report has the following structure.
Section~\ref{sec:Goals}, contains a description of all the goals that has been set up for this project.

Secondly all the milestone for the different subsystems to be developed are located in section~\ref{sec:Milestones}.
Next section, section~\ref{sec:Requirements} lists all the specified requirements.

Section~\ref{sec:Planning}. The planning section shows the overall plan for this semester in tableform. For a more detailed description of the plan, open the file Swarm\_BIS.pod, which is located in the same dropbox map as this report.

Section~\ref{sec:NaiadOldSystem} contains an introduction of the subsystems that has been investigated during this year on the old Naiad platform. As well as an Quick starting guide for the Naiad.

Next section, section~\ref{sec:Naiadsister} describes the new platform that has starting to take form. In this section one can find the documentation of the new IMU card as well as future work on the platform.

Section~\ref{sec:WirelessCommunication} describes the need for a WiFi solution, what has happen on the wireless communication front this year, a short description on the configuration is also available and lastly there are behavioral-issues and future work.

Section~\ref{sec:Underwatercommunication} Underwater communication is described, this section is divided into two parts. Firstly there is the a section to del with the electronics that has been implemented. Secondly the communication between two computers using the developed electronic system solution.

Section~\ref{sec:Sonarcommunication} contains informations information on which students that will work on this during there masters thesis.

The swarm behavior section, section~\ref{sec:Swarmbehavior} is divided as dollows. A short introduction about what has been done and the purpose for this year. Secondly, it contains a description on the codes that has been implemented for the klobots. Thirdly a description of the Matlab code for the self-organizing coordinate system and the different start up codes in matlab for the next years students. Fourthly a summery of the S-DASH algorithm that still needs to be implemented is provided. and lastly  conclusions and future work.

Lastly the Section~\ref{App:appendix} contains all appendices for this project.


In this report the author's name for every subsection can be found in () after the subtitle. The author is responsible for the text until next name appear. The purpose of this is to make grading easier for the examining teacher.
The project leader, Peter Cederblad has putting it all together.