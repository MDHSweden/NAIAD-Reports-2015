\documentclass[12pt,a4paper]{article}
\usepackage[utf8]{inputenc}
\usepackage{amsmath}
\usepackage{amsfonts}
\usepackage{amssymb}
\author{Weronica Kovala}
\title{Lessons learned - Electronics}
\author{Anette Hilmersson, Lennie Carlén Eriksson, Patrik Pärlefjord}    					%Enter your name here                                

\setlength{\footskip}{7ex}
\begin{document}
\maketitle
\section{Time just disapears}
When the project starts one think one has all the time in the world. You do not. Before you get a desk the first period is over. Before you even had time to design a card the second is over and if you have time you might be able to solder your first prototype before the report is due. Ok this might be a slight over exaggeration but with time for milling, waiting for parts and just coming up with a new idea for your first idea doesn't work (it won't) it kind of feels like this. 

\section{If you are taking over someone else's work}
One thing we were a bit naive of in the beginning was that if we could use it it was enough. But slowly we begun encountering more and more problems because we did not understand the card given to us. With next to no documentation allot of work had to be done to 'recover' missing information by testing the cards and looking at what we have. 

\section{Documentation}
Documenting clearly is hard, and can be boring but it is important to keep in mind throughout the project that someone else should be able to get your files and notes and understand what you are doing. 

It can be a good idea to weekly or every two week write down a status report. We had this as a totally optional ongoing documentation. Meaning most did not do any wile others wrote everyday. There was no organised system. It can be good to be able to back trace when an idea came up or when one begun with something to detect if someone or some project is lacking. 

\section{Communication}
Like most group will have written, communication both in the group and with other groups in the project in crucial. Misunderstandings are just to easy and create allot of double work for all involved. It also easily creates disputes in the group. Working under pressure in a new environment where everyone have their own idea on how things should be done can make people angry about nothing when things do not go as planned. Especially in the beginning were there allot of arguments on the work environment. Who should sit where, how much noise was ok to make, who should sit on what chair and so on. When you read it like this it is understandably that you discard this as us being incredibly childish, but going back to several strong wills and pressure with no real leader (read teacher/authority figure) it comes down to this. 

\section{Don't re-invent the wheel}
You're not the best engineer in the world. Someone, somewhere has probably already tried every idea that pops up in your head. Even though the trial and error approach is more fun, you might be able to save both time and money by finding what has, and more importantly, hasn't worked for other people.

\end{document}