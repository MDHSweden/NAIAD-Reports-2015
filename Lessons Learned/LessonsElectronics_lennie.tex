\documentclass[12pt,a4paper]{article}
%\usepackage[utf8]{inputenc}
\usepackage{amsmath}
\usepackage{amsfonts}
\usepackage{amssymb}
\usepackage[swedish]{babel}
\author{Weronica Kovala}
\title{Lessons learned - Electronics}
\author{Anette Hilmersson, Lennie Carl\'en Eriksson, Patrik P\"{a}rlefjord}    					%Enter your name here                                

\setlength{\footskip}{7ex}
\begin{document}
\maketitle
\section{Time just disapears}
At the start of the project one might think that you have all the time in the world. You do not. The first period will be over before you even get a desk. Before you had time to design a card the second period is over and if you have time you might be able to solder your first prototype before the report is due. This might be a over exaggeration but when counting the time for milling, waiting for parts and just coming up with a new idea when your first idea doesn't work (it won't), then this is what it feels like. 

\section{If you are taking over someone else's work}
At the beginning we were a bit naive. We thought that all the existing electronics was enough for what it was made for. But slowly we began to realise we had to redo some of the work the previous group did. When we tried to redo the electronics we realised it would take a long time to understand all the existing circuits, since almost none of the boards were documented. So a lot of work had to be done to backtrack the designs and document them before changes could be made. So we learned the importance of documentation the hard way. 

\section{Documentation}
Documenting clearly is hard, it can also be boring, but it is important to keep in mind that someone else might want to read your notes and files and understand what you are doing. 

It can be a good idea to write down a status report every week or two. We decided to make an ongoing documentation optional. Meaning some did not do any while others wrote something everyday. There was no organised system. Ongoing documentation might be a good thing since one is able to see when an idea came up or when one began working with something. This will help to detect if someone or some project is lacking. 

\section{Communication}
Like most groups will have written, communication both in the group and with other groups is crucial. Misunderstandings are easy to make and create a lot of work for all involved. It might also create disputes in the group. Working under pressure in a new environment where everyone have their own idea on how things should be done can make people angry about nothing when things do not go as planned. Especially in the beginning when there are a lot of arguments regarding the work environment. Who should sit where, how much noise is ok to make, who should sit on what chair and so on. When you read it like this it is understandably that you discard this as being incredibly childish, but going back to several strong wills and pressure with no real leader (read teacher/authority figure) it comes down to this. 

\section{Don't re-invent the wheel}
You're not the best engineer in the world. Someone, somewhere has probably already tried every idea that pops up in your head. Even though the trial and error approach is more fun, you might be able to save both time and money by finding what has, and more importantly, hasn't worked for other people.

\end{document}
