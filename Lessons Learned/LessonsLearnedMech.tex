\documentclass[11pt,a4paper]{amsart}
\usepackage[a4paper,vmargin={20mm,20mm},hmargin={20mm,20mm}]{geometry}
\usepackage[swedish]{babel}
\usepackage[utf8]{inputenc}
\usepackage[pdfborder={0 0 0},colorlinks=true, urlcolor=blue, citecolor=red, bookmarks=false]{hyperref}
\usepackage{fancyhdr}
\usepackage{multicol}
\pagestyle{fancy}


\rhead{NAIAD}
\lhead{Martina \"{O}hlund}


\title{Lessons Learned - Mechanics}
\author{[Martina \"{O}hlund, Ande Hana]}    					%Enter your name here                                

\begin{document}
\begin{multicols}{2}

\thispagestyle{empty}
\maketitle

%\section*{Guidelines}
%The report should be typeset with \LaTeX !
%The requirements that are set for the different grades are:
%\begin{description}
%\item[Grade 3] Be there and do a good job
%\item[Grade 4] Be there and take initiatives to fulfil the goals
%\item[Grade 5] Be there, take initiatives to fulfil the goals and suggest (close to research) new approaches    
%\end{description}

\section{\bf{Introduction}}
\noindent We have been in the mechanics group, where we were responsible for designing, constructing and maintaining Naiad. This paper describes what we have learned during this project and what we would like to share to future developers of Naiad as to not make the same mistakes. 

\section{\bf{Communication}}
\noindent Communication is very important when working in a group. It is essential that the communication both within your own group and between the other groups in the project works. You do not want to spend a large amount of time on e.g. designing and constructing a complete part or system only to find out it is useless for the project. Therefore, make sure to find out what demands the other groups have and do not keep things to yourself. 

This works the other way around too, do not assume that something you need done from another group is being worked on unless you ask the responsible person about it and make sure they understand what you want. They may have misinterpreted you or not listened the first time it was brought up. 

\section{\bf{Test your designs and be creative}}
If you have resources for making prototypes, then you should continuously make prototypes and test your design. This helps discovering flaws and aspects you may not have even thought about. 
We were blessed (or cursed) with a 3D-printer, so we made sure to always test our different designs to see what could be improved. 

However, do not trust blindly on the equipment you have available. We encountered countless of issues with the 3D-printer. For some models we had to think about other options, for example if it could be milled instead or if we should spend money on making a prototype in a 3D-printer with much better precision. 

Be creative with what you have. For example, if you cannot find the right tool for some assembling, then find something else that could do the work. Do not wait around until you have received the right equipment as this could take a long time. 

Also, when designing a part, try to think as simple as possible and avoid having hard-to-get components in the design. It can take a long time to get a hold on non-standard materials or tools etc.

\section{\bf{Be critical}}
\noindent Be critical about every idea and solution. Can a design be improved, does it work with the surrounding system or do you have to come up with a new design? 

Also, make sure to always discuss your ideas or solutions with others, as they bring a new perspective to the equation. This can really help saving a lot of time. 

\section{\bf{Mistakes will be made}}
\noindent Plan to make mistakes. It is important to try your best to meet deadlines, so always add some extra time when setting the deadlines. Be pessimistic and assume you will encounter problems, because you will. 

\section{\bf{Planning}}
\noindent Make sure to plan your work. A good idea is to also set up deadlines within your group for smaller things to be done. For example, break apart a big deadline into smaller deadlines so that you continuously work on the piece. Else, there is a risk that you only look at the larger deadlines and the work that needs to be done for that may seem much more than it actually is. Also, as people are prone to procrastination, it really helps with smaller deadlines so that you always keep moving forward and have time to detect mistakes or bad designs. 

If you need to use machines or equipment which has a high demand, for example the mills, then make sure to plan in time and learn the machines properly. The mills are very often full booked and it is hard to find a time to use them. You do not want to spend the time you do get on them making easy mistakes.  
\end{multicols}
\end{document}  