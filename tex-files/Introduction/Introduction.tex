\section{Introduction}
\noindent
This is a report of project Naiad. The name Naiad comes from the Greek mythology where the Naiads were water nymphs that controlled different bodies of fresh water\cite{naiadwiki}. Somewhat like the Greek Naiads this Autonomous Underwater Vehicle (AUV) will work towards a cleaner and safer underwater environment, both on its own and together with other Naiads as a group. 

The long term goal for Naiad is for it to screen, map and maybe even clean the bottom of the Baltic sea from toxic and in other ways hazardous waist. The seas has been seen as a dumping ground for all sorts of things for decades. 23 000 barrels, containing 9 tons of mercury, is buried in the seabed outside of Sundsvall, Sweden \cite{mercury_SR}. 

The short term goal is to enter RoboSub 2015. To do this the AUV has to be able to do easy missions based on vision and sensors, using different tools such as torpedoes, markers and a manipulator of some sort. The project members has worked in different groups aiming towards one common goal and the results are a functioning autonomous robot that can complete simple missions. 

During the length of this project, the robot Naiad has been developed. The result of this is an autonomous underwater robot which is able to solve simple tasks it is given. During the project electronics, such as solutions for implementing a fibre optical gyro (FOG) and a remote control, has been developed. A node based communication system for communicating on the Controller Area Network (CAN) bus has been created by the software team. The mechanics team has developed the wings and the front of the toolplate, both of which has been manifactured. 