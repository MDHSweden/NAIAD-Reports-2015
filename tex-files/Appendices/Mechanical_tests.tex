\section{Mechanic tests}
Following is a series of tests that should be performed on the mechanical parts of the system. 

\subsection{Testing the o-ring}
\label{oring}
\subsubsection*{Purpouse of the test case}
The purpouse of this test is to make sure that there are no dents in the o-ring, which would make it unsafe to use under water. 
\subsubsection*{Description}
For this test a person will feel through the o-ring to make sure that there are no dents in it. 
\subsubsection*{Resources}
O-ring, petroleum jelly
\subsubsection*{Preconditions}
The preconditions are that the o-ring is dry and grease free. 
\subsubsection*{Post conditions}
After preforming this test the o-ring should be greased and for a pass it should have no dents. Any dents will count as a fail in this test. 
\subsubsection*{Flow of events}
For this test the person performing the test should have some petroleum jelly on their fingers and gently feel every piece of the o-ring to make sure that there are no dents in it. Make sure that the entire o-ring is properly greased. 

\begin{tabular}{| l | c |}
\hline
Check point & Measured \\ \hline
Check for dents &  \\ \hline
\end{tabular}
\subsubsection*{Inclusion/Exclusion points}
The test will count as a pass if the o-ring is free of dents and can be completely greased. All other cases will count as a fail. 
\subsubsection*{Special requirements}
There are no special requirements for this test. 

\subsection{Closing of the hull}
\label{Hulltest}
\subsubsection*{Purpouse of the test case}
This test is supposed to make sure that the hull is water proof when closed. 
\subsubsection*{Description}
The test is done to make sure that the lid is closed properly to make sure there is no leakage when the AUV is put into water. 
\subsubsection*{Resources}
Required resources for this test is the hull of Naiad, some paper, petroleum jelly, a long rope, and weights (for example in a weight belt for diving). 
\subsubsection*{Preconditions}
There are no preconditions for this test. 
\subsubsection*{Post conditions}
After preforming this test the lid of the hull should be closed and the hull should be waterproof. 
\subsubsection*{Flow of events}
Firstly, the hull should be fitted with either the electronics, for an actual run with the robot, or with paper for testing the waterproofness of the hull. 

Then the o-ring should be tested, this is done through feeling the o-ring all the way around using petroleum jelly to grease it thoroughly, for a closer description of this see \ref{oring}. Then it should be placed in the o-ring slot which before hand should be greased as well. Then the lid can be placed and closed to ensure waterproofing. When placing the lid, make sure that no cables are being pinched between the hull and the lid. Also make sure to place the lid from straight above so that the o-ring stays in place. 

\begin{tabular}{| l | c |}
\hline
Check point & Measured \\ \hline
Test of o-ring &  \\ \hline
Placement of o-ring &  \\ \hline
Placement of lid &  \\ \hline
Cable check & \\ \hline
\end{tabular} 
\subsubsection*{Inclusion/Exclusion points}
This test will not test which pressure the hull can sustain, it will just test that there will not be any immediate leakage. 
\subsubsection*{Special requirements}
There are no special requirements for this test. 