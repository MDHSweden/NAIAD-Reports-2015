\subsection{Communication} % Jonatan & Hampus

\noindent All of the node to node communication is based on either CAN or Transmission Control Protocol (TCP) where the former is communication between CAN-cards and the later is between nodes that are either on a BBB, a GIMME-2 card or other nodes that are connected via Ethernet. A node on the CAN-bus is still able to communicate with a node on a BBB by passing through a translation node.

\subsubsection{Node based TCP communication} % Hampus
\noindent All of the TCP communication is routed by a central server node that will relay messages to the desired destination node which is specified in the message itself. Having a central node where all the messages pass through adds a lot of overhead but it also adds a lot of flexibility and simplicity for a user. It allows each client to only require a connection to the server while still being able to communicate with all of the connected nodes. Furthermore, this makes it easy to add more nodes to the system with less effort as the other nodes does not have to establish a new connection, they only have to send a new type of message. The center node can also easier track which nodes are alive and connected rather then track every single cross connection between clients.

\subsubsection{Serial Peripheral Interface} % Hampus
\noindent The SPI on Naiad is using the standard master/slave configuration, where CAN-cards are acting as masters and the attached peripherals are the slaves. There are two peripherals that are using SPI in Naiad, an LCD display and an ADC that is connected to a fiber optic gyroscope\cite{fog}. The highest bit rate possible for SPI on the CAN-cards are 8Mbit per sec as the clock rate of the CAN-card oscillator is 16MHz. The high bit rate is needed as it allows faster data sampling from the fiber optic gyroscope for a better precision when using the data for integration.

\subsubsection{Universal Asynchronous Receiver/Transmitter} % Hampus
\noindent Universal asynchronous receiver/transmitter (UART) is the communication link between CAN and TCP where a CAN-card and a BBB is connected. This link have the lowest communication bit rate which in turn sets the limit of the combined bit rate to and from all of the sensor cards. While the BBB support CAN, it does not support the high voltage that is on the bus which is why UART is used instead. Both the inertial measurement unit (IMU) and the side scan sonar is using UART to send the raw sensor data. The IMU is connected to a CAN-card and the side scan sonar is connected directly to a BBB due to the high amount of raw data it produces. The bit rate for all UART communication is 115200 bits per second.

\subsubsection{Controller Area Network} % Hampus
\noindent The Controller Area Network (CAN) protocol is fully hardware supported on the microcontroller AT90CAN128, supporting the standard CAN controller 2.0A \& 2.0B. Advantages of using CAN is the ease of adding additional nodes and the fact that it is a broadcast protocol. This allows a very modular system where new nodes can enter the bus without the need of establishing connections to already existing nodes. One drawback of CAN is the limited data bandwidth. To reduce the load on the CAN-bus, many calculations are done directly on the CAN-cards, so the data sent is directly usable.
