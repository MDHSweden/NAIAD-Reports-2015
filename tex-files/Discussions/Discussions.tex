\section{Discussion}
\noindent
The documentation of the work previously done in this project was not that well done, some parts were missing and some parts had faults in them. This made it harder for the team to move forward in their efforts to finish the project. Since the project was already started, the team members agreed that the best way was to continue with what already started. Restarting would have taken too much time. However as the project progressed, it was clear that some parts were to undocumented, others did not reach the right standard and had to be dismissed for that reason. This is to be seen as one of the main reasons as to why the project could not be finished on time. Not all parts had to be redone, some was harder to change, such as the hull which would have taken a lot of money and a lot of time to redo. Some parts were also fully functioning, which made it unnecessary to change them. 

The reason why the umbilical cord was never implemented was that a sponsor withdrew at the last minute and since the cord was too expencive to buy, it was put on hold. However, with the added functionality of the remote, the need for an umbilical cord which is easy to connect and disconnect is reduced, since some of the communication which should have been done through this is instead done via the remote control. An advantage of actually having an umbilical cord would have been the possibility to run Naiad on external power rather than batteries. 

Since the system is currently specified for some of the RoboSub 2014 missions, this will have to be changed before the competition. However, there are only minor changes to the missions which means this will be an easy task. 

Working in a larger group with shared documents have caused problems. Almost all documents was accidentally erased twice, only being able to be retrieved thanks to a computer not updated the folder. Working as a group there is also a possibility that more then one person is working in the same document at the same time. If saved by one this will overwrite the others. This have happened way more than once. For these reasons a better system then Dropbox should be used for sharing files.

The previous Naiad team had one person who was working full time on marketing and getting sponsors for the project. Since there was only 12 persons in this project group, it was decided that there was to many other things to do. This responsibility was supposed to be shared between all members of the group. However in the end, it was only the project manager working on this. As the project progressed it was apparent that the project would have benefited from having one person working full time on this since dedicated efforts are needed to get companies to move forward with sponsoring requests. 

\subsection{Mechanics discussion}
\noindent
The mechanics group is the group that have had to mostly adapt to what was given. This has both been a blessing and a curse. Some decisions on design was already complete meaning some workaround was necessary. At the same time a as it has been good to have a base to start from. Because of time and cost in manufacturing all of the constructed pieces from last year had to be kept. For the most part the designers from last year hat thought abut how to attach the missing pieces. One place where this was not taken in to account was the wings. There are no way to attach the wings into the top of naiad at the beginning. Also the thrusters needs to be graced aver every-time they have been in water. Meaning access to them is necessary. This complicated the design of the wings.

The tool-plate is milled just like the rest of the hull. Milling however eats allot of time. It is hard to get time on the milling machine in Eskilstuna. Every time someone else had used the machine it needs to be recalibrated which also ate a lot of time. To reduce the amount of people wanting time on the machine as well as the complexity on some parts drove the mechanics group to learn the biggest milling machine available in Eskilstuna. This machine has a steeper learning curve than the other machines. Despite the extra time to learn it the conclusion is that is was the right choice. 

The wings and some other smaller parts of the parts produced this year was printed in the schools new 3D-printer. This have not always been without frustration. It is not a high quality printer and sometimes several prints needed to be made before a satisfactory print was done. The printer material also did not seam to go with the chlorine in the pool we were testing the robot in. It is believed that a layer of paint will fix this. 

Another frustration for the mechanics group is the vague description or waiting in response on how things should be mounted by other groups. This is something that is always a problem in all group assignment but because we only had one shot on most of the mechanic pieces it was extra important to get it all right the first time. 

\subsection{Electronics discussion}
\noindent
The electronics groups progress went really smoothly in the beginning. The cards from last year was already produced and add-ons seamed easy. However as the project progressed, the more important it was, to not only be able to use the cards from last year, but also understand them. Most of the documents regarding electronics from last year was sub-standard or missing completely therefore work on recreating this documentation have been done as well, taking up surprisingly large amount of time.

The electronics group have produced allot but as always when constructing hardware it takes longer time than expected and there had been an hope to reach further that we have done. For example have time to order final version cards from Würth and ideally even have time to produce these cards. This have not been the case. All cards have been prototyped and tested but further works needs to be done on most cards. 

\subsection{Software discussion}
\noindent
The software group has worked well together reaching most of the primary goals, achieving a reliable base for future work.

Documentation is just as important as the work you document on in projects of this magnitude. The documentation from the first Naiad team was hard to follow, this led to that almost all of their work had to be redone.
This year the software team has focused on following a standard that should be easy to use and using at most 3-5 layers of code in libraries to avoid having to spend much time understanding where values and functions descend from.

By combining the previous firmware from Naiad and from Vasa, a well working firmware has been achieved in short time, there is still some issues, but overall it works good.

From the start the GIMME-2 boards was supposed to replace BBB but due to limitations of the operative system running on the GIMME-2 boards this would not be an effective alternative.

A lot of problems hard to explain have occurred in the CAN-cards. Whether it is Ada or the firmware that causes it, the persons working with have tried to figure out but without any real success. This are not logical issues, but rather for example crashing when leaving a procedure, variables changing values without reasons etc.
