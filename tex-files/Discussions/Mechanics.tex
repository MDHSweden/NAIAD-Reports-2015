\subsection{Mechanics discussion}
The mechanics group is the group that have had to mostly adapt to what was given. This has both been a blessing and a curse. Some decisions on design was already complete meaning some workaround was necessary. At the same time as it has been good to have a base to start from. Because of time and cost in manufacturing all of the constructed pieces from previous group needed to be kept. For the most part the designers from last year had thought about how to attach the missing pieces. One place where this was not taken in to account was the wings. There were no way to attach the wings into the top of Naiad at the beginning. Also the motors needs to be greased after every-time they have been in water, meaning access to them is necessary. This complicated the design of the wings.
The tool-plate is milled just like the rest of the hull. Milling however is time consuming. It is hard to get time on the milling machine in Eskilstuna. Every time someone else have been on the machine it needs to be recalibrated witch also ate allot of time. To reduce the amount of people wanting time on the machine as well as the complexity on some parts drove the mechanics group to learn the biggest milling machine available in Eskilstuna. This machine has a steeper learning curve than the other machines. Despite the extra time to learn it, it was the right choice. 
The wings and some other smaller parts produced this year was printed in the schools new 3D printer. This have not always been without frustration. It is not a high quality printer and sometimes several prints needed to be made before a satisfactory print was done. The printer material also there were some concerns with chlorine in the pool where we were testing the robot. It is believed that a layer of paint will fix this. 
Another frustration for the mechanics group is the vague description or waiting in response on how things should be mounted by other groups. This is something that always is a problem in all group assignment but because we only had one shot on most of the mechanic pieces it was extra important to get it all right the first time. 

