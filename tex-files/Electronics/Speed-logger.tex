\subsection{Speed-logger} %Patrik
\noindent
The purpose and mechanical design of the speed logger can be found in section \ref{SL_mec}. The electronics parts of it is divided to an outer sensor and an inner analysing circuit.

\subsubsection{Sensor}
\noindent
The sensor is based on two Allegro A1301 hall sensors. They translates differences in the magnetic field to analogue voltages. These are placed side by side with 5 mm spacing on a small PCB near the rotating turbine. 
The propeller has magnets mounted on its blades, and by reading the timing differences between the sensors, rotational direction and speed can be determined.

\subsubsection{Circuit}
\noindent
The signal from from the sensors needs to be altered in order to work as an interrupt edge for the responsible CAN-card.
The IC circuits on the sensor has a default voltage of half of the supply voltage.
To get an accurate reference voltage, a third A1301 sensor is used inside the hull as a reference voltage.
The difference between the inside and outside sensors is amplified in order to trigger the interrupt on the MCU. 

	
	
